\chapter{Conclusion} \label{sec:conclusion}

This thesis presented Substrate, a new framework for analog generators
written in the Rust programming language, as well as SRAM22, a configurable
SRAM generator built on top of Substrate.

\section{Additional Features}

Substrate has many features beyond the ones described in this thesis.
In particular, Substrate provides APIs for functional verification
and for verifying digital timing constraints (eg. setup/hold times)
in contexts where analog workflows are more convenient.
The latter feature was used to design and verify a schematic-level tree serializer
generator intended for die-to-die links. Substrate is also capable
of generating power straps compatible with Hammer, the VLSI
flow tool developed at Berkeley \cite{hammer}.

\section{Future Work}

There are a wide variety of directions for future work.
We suggest a few here.

\begin{itemize}
\item Designing an intermediate representation for representing circuits.
Such an intermediate representation would be useful to enable faster
design space exploration, and could enable greater interoperability between generator frameworks.
\item Smarter algorithms for analog routing. In particular, it would be nice to have 
an automatic router capable of generating symmetric/matched differential routing,
while understanding which nets are aggressors and which are victims.
\item Easing the process of integrating analog IP into digital flows, a process
that currently requires setting up flows for several tools not often used in a purely-analog workflow.
Instantiating macros in digital flows is usually done outside of the generator framework,
but there are benefits to having the generator be aware of how it integrates into a digital-top floorplan.
Power strap consistency is one such example.
Utilities for quickly setting up analog/digital co-simulation would also make the process of analog
integration easier.
\end{itemize}
