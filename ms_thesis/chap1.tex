\chapter{Introduction}

\section{Motivation}

Chip design is expensive, with NRE costs in the ballpark of millions to
hundreds of millions of dollars per chip \cite{horowitz}.
With ever-changing process nodes, porting IP blocks and tool configurations
across processes becomes a painstaking endeavor.
While digital circuits can be produced mostly automatically from RTL,
no such procedure for analog circuits is in widespread use.
The result is that analog blocks are, by and large, designed by hand,
laid out by hand, and verified by hand, with limited reuse of methodology.
Furthermore, only a small portion of analog design time is spent designing
truly custom blocks. The majority is spent designing and integrating
mostly standard blocks, such as ADCs, PLLs, and LDOs.

Generators have emerged as a solution to the problem of encoding design
methodology and enabling greater automation and reuse \cite{align, template-driven-analog}.
Generator frameworks allow users to write code that accepts some set of parameters
and produces instances of a circuit. While writing a programmatic generator
may sound good in principle, there are almost always times when it is simply
faster to design or lay out a circuit by hand. This is especially the case
when designing circuits that exploit process-specific devices or design rules.
Thus, a good generator framework must make it easy to incorporate designs
produced outside the framework.

The Berkeley Analog Generator (BAG) is one generator framework developed
at UC Berkeley \cite{bag}. Although BAG has been used to design a variety
of circuits \cite{serdes-gen}, it has some limitations.
BAG is closely intertwined with Cadence Virtuoso, and although there
are ongoing efforts to remove this dependency, BAG generators still require
Virtuoso, at least for initial setup. For open-source processes such as the
Skywater 130nm process, the requirement to have access to Virtuoso
makes it difficult to write a standalone generator.
Additionally, BAG provides relatively few primitives for generating
area-constrained layouts. The default transistor cells provided by BAG
PDK plugins are difficult to customize, since BAG expects them to have
certain properties (eg. particular gate/source/drain pitches). Fitting
the existing BAG template cells into pitch-constrained layout environments
is not always possible.
Another limitation is in routing flexibility. BAG routes are usually specified
by assigning nets to track indexes via a YAML file. BAG does not support
automatic signal routing.
Finally, the use of Python as a language for writing generators precludes strict type-checking,
leading to runtime errors that could easily have been prevented at compile time.
Python also suffers from low performance, meaning that performance-sensitive algorithms
must be written in another language and then wrapped in a Python API. This makes
writing flexible layout algorithms much more tedious.

As a result, we believe that there is room for significant improvements in
analog generator workflows. 

\section{Thesis Organization}

Chapter \ref{sec:substrate} provides a high-level overview of Substrate, a framework
for writing analog/mixed-signal generators.
Chapter \ref{sec:sram22} describes SRAM22, an open-source configurable SRAM generator
build on Substrate. Chapter \ref{sec:conclusion} provides a conclusion and suggests
future directions of work.
